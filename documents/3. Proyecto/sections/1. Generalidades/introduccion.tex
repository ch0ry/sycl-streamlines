\section*{Introducción}
\addcontentsline{toc}{section}{Introducción}

% 1. Origen de la computación autonómica (Citation needed for industry tendencies)

La computación autonómica, concebida inicialmente por IBM en el año 2001, se refiere al uso de sistemas auto-gestionados con la capacidad de operar y adaptarse, o en lo posible, sin la intervención de un ser humano. En este sentido, este acercamiento tiene como objetivo la creación de sistemas computacionales capaces de reconfigurarse en respuesta a cambios en las condiciones del entorno de ejecución al igual que los objetivos del negocio \cite{horn_2001}.

Esta autonomía es adquirida con el uso de ciclos de control, en el caso de la computación autonómica, de los ciclos más populares es el ciclo MAPE-K, sigla de \textit{Monitor Analyze Plan Execute - Knowledge} \cite{Arcaini_2015}. Estos le dan la capacidad al sistema de monitorear tanto su estado actual como el entorno en el que este se encuentra, analizar la información recolectada para luego planear y ejecutar los cambios requeridos sobre el sistema a partir de una base de conocimiento \cite{RutanenKalle2018McoO}.

Una de las áreas que pueden verse especialmente beneficiadas de la computación autonómica es la del internet de las cosas (IoT). Algunos de estos aspectos están relacionados con la heterogeneidad de estos dispositivos, en cuanto a marcas, protocolos y características; la dinamicidad, en cuanto al movimiento que estos presentan entre entornos de ejecución o incluso una desconexión; al igual que la distribución geográfica de estos lo cual dificulta la intervención directa sobre ellos \cite{Tahir_2019}. 

Esto puede verse el Smart Campus UIS, una plataforma IoT de la Universidad Industrial de Santander, que permite usar dispositivos con el fin de monitorear y recolectar de información en tiempo real con el objetivo de apoyar la toma de decisiones, mejora de servicios, entre otros \cite{henry_2020}.

Ahora, esta plataforma ha tenido esfuerzos en el desarrollo de características propias de un sistema autonómico. Uno de estos ha sido la integración de mecanismos para la auto-descripción de la arquitectura desplegada en un momento dado \cite{msc_henry_2022}. En el contexto de la computación autonómica, esta capacidad hace parte del monitoreo dentro del ciclo de control. 

Partiendo de lo anterior, y con la intención de dar continuidad a los esfuerzos de desarrollo realizados en Smart Campus UIS, se plantea como caso de estudio, a partir de las capacidades de auto-descripción que la plataforma, se pretende el proveer a esta la capacidad de auto-configuración a partir de un conjunto de mecanismos de adaptación que permitan, desde la definición de un objetivo a lograr, la alteración de la arquitectura desplegada. 