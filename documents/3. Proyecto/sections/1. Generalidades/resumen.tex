\section*{Resumen}

\textbf{Título:} \tit\footnote{Trabajo de Investigación}

\textbf{Autor:} \nam\footnote{\fac.\ \esc.\\ Director: \tdir\ \dir, Codirector: \tcdir\ \cdir }

\textbf{Palabras clave:} { Computación Autonómica, Internet de las cosas, Adaptabilidad, \\Arquitecturas Software, Lenguajes de Descripción de Arquitecturas, Smart Campus }

\textbf{Descripción:}

La complejidad de los sistemas computacionales tiene origen en diversos factores. El aumento de la cantidad de dispositivos que los componen; junto con las condiciones variables de sus entornos de ejecución; dificultan la administración de los sistemas computacionales. Algunos de los sistemas que se ven afectados por esta complejidad, suelen ser los sistemas IoT, entre ellos los conocidos como Smart Campus. En este trabajo de investigación, se propone una implementación, para la plataforma IoT Smart Campus UIS, basada en los principios de computación autonómica, con el fin de reducir la complejidad de la administración de las aplicaciones desarrollándose sobre esta. Para lograrlo, se inicia definiendo una notación que permite la declaración de las arquitecturas objetivo; segundo, se diseña e implementa un mecanismo responsable del monitoreo de la plataforma al igual una manera de establecer las diferencias entre el estado objetivo y el de referencia; Posteriormente, se implementa una serie de mecanismos para modificar el estado actual de la arquitectura; finalmente, se lleva a cabo la validación de la implementación para evaluar su capacidad para reducir las diferencias entre el estado objetivo y el estado presente.

\newpage

\section*{Abstract}

\textbf{Title:} {Self-adaptive software architecture mechanisms for the Smart Campus\\UIS platform}\footnote{Research Work}

\textbf{Autor:} \nam\footnote{Faculty of Physics-Mechanics Engineering. School of Systems Engineering and Computer
Science.\\ Advisor: \tdir\ \dir, Co-Advisor: \tcdir\ \cdir }

\textbf{Palabras clave:} { Autonomic Computing, Internet of Things, Adaptability, \\Software Architecture, Architecture Description Languages, Smart Campus }

\textbf{Descripción:}

The complexity of computational systems originates from various factors. The increase in the number of devices that compose them, along with the variable conditions of their execution environments, makes the administration of computational systems challenging. Some of the systems affected by this complexity are often IoT systems, including those known as Smart Campus. In this research work, an implementation is proposed for the IoT Smart Campus UIS platform based on the principles of autonomic computing, aiming to reduce the complexity of managing applications developed on it. To achieve this, it begins by defining a notation that allows the declaration of target architectures. Second, a mechanism for monitoring the platform is designed and implemented, along with a way to establish differences between the target state and the reference state. Subsequently, a series of mechanisms are implemented to modify the current state of the architecture. Finally, the implementation is validated to assess its ability to reduce differences between the target state and the present state.

\newpage