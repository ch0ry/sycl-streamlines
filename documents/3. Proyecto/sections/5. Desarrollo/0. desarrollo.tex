\section{Validación de la Implementación} \label{sec:experimental}

Lo último a realizar fue la validación de las funcionalidades de la aplicación desarrollada. Para ello, se establecieron escenarios de prueba por los cuales se pueda comprobar el funcionamiento de aplicación. Estos, se definieron con el fin de probar la capacidad y la efectividad de los mecanismos implementados de modificar el estado inicial de la aplicación hacia el estado de referencia, al igual que la identificación de algunas de las falencias de esta.

Ahora, se debe resaltar que, las pruebas se realizaron usando un servicio genérico, apodado Mocker, encargado de la generación de datos, especialmente desarrollado para funcionar en condiciones ideales. 

Para cada una de estas pruebas, se definió un estado de referencia, dadas unas condiciones iniciales, y una serie de eventos los cuales modificarán nuevamente el estado de la aplicación, las cuales tendrán que ser nuevamente manejadas.

\subsection{Escenario Experimental A} \label{sec:EscenarioExperimentalA}

El primer escenario, busca el evaluar la capacidad de la implementación realizada de iniciar toda una aplicación desde cero. Partiendo de esto, se definió la aplicación vista en la figura \ref{fig:ExpA}. Esta tiene dos locaciones raíz, con requerimientos de datos iguales en ambas.

\begin{figure}[H]
    \centering
    \caption{\\Diagrama del escenario experimental A}
    \label{fig:ExpA}
    \includegraphics[width=0.8\linewidth]{images/ScenarioA.pdf}
    \vspace{-4mm}
\end{figure}

Inicialmente, no habrán dispositivos presentes en ejecución, por lo que el sistema tendrá que identificar la falla, definir las acciones de adición correspondientes, y ejecutarlas para poder suplir con los requerimientos de datos.

Para ello, tomando como referencia la figura \ref{fig:ExpA}, se realizó la declaración del estado de referencia, presente en el anexo \ref{ape:ExpA}, y, como observa en la figura \ref{fig:ValidA}, se validó usando \textit{Lexical}.

\begin{figure}[H]
    \centering
    \caption{\\Validación de la declaración de referencia realizada para el escenario experimental A}
    \label{fig:ValidA}
    \includegraphics[width=0.7\linewidth]{images/ValidationLexicalA.png}
    \vspace{-4mm}
\end{figure}

Ya con las condiciones iniciales, y el estado de referencia definido, se estructuraron, las directivas, vistas en el anexo \ref{ape:DirectivesA}, que permitirían realizar la adaptación de la arquitectura hacia el estado de objetivo. 

Partiendo de todo lo anterior, se definió la serie de pasos a realizar para este primer escenario experimental:

\begin{enumerate}[itemsep=0mm]
    \item Desplegar los servicios de Smart Campus UIS
    \item Desplegar los servicios \textit{Bran} y \textit{DoThing}.
    \item Usando \textit{Lexical}, establecer el estado de referencia de la aplicación.
    \item Declarar en los endpoints de \textit{Bran}, las directivas definidas.
    \item Desplegar el servicio \textit{Looker} para la aplicación.
    \item A la nivelación del estado de la aplicación.
\end{enumerate}

Al final de la simulación de este escenario, se eliminarán todos los servicios desplegados con el fin de asegurar una ejecución limpia de futuros procesos.


\subsection{Resultados De Escenario Experimental B}

El objetivo principal del escenario experimental B, era evaluar la capacidad de la implementación realizada de recuperar una aplicación en estado de coherencia, en donde se presenta una falla masiva, nuevamente a un estado válido; al igual que la capacidad de Bran de manejar más de una aplicación, como se había nombrado en la sección \ref{sec:Centering}.

Igual que durante el escenario experimental A, se realizó el despliegue de los servicios \textit{Bran} y \textit{DoThing}. Como se observa en la figura \ref{fig:Bran2Apps}, se definieron las arquitecturas objetivo de las dos aplicaciones a monitorear, al igual que las directivas a usar durante este proceso.

\begin{figure}[ht]
    \centering
    \caption{\\Aplicaciones y directivas registradas en Bran}
    \label{fig:Bran2Apps}
    \includegraphics[width=0.9\linewidth]{images/BranBScenarioDeclarition.png}
    \vspace{-4mm}
\end{figure}

Seguidamente, como se observa en la figura \ref{fig:Looker2Apps}, se desplegaron los servicios de \textit{Looker} para cada una de las aplicaciones a monitorear. Ya con esta base lista, y como se definió en los pasos a seguir, se desplegaron de manera manual, un total de 3 servicios \textit{Mocker} con configuraciones específicas (tanto de los datos que estos reportaban, como de los tiempos de reporte) con el fin de cumplir con los requerimientos de datos definidos para las dos aplicaciones. 

\begin{figure}[H]
    \centering
    \caption{\\Dos instancias de looker monitoreando las aplicaciones declaradas}
    \label{fig:Looker2Apps}
    \includegraphics[width=0.9\linewidth]{images/Looker2Apps.png}
    \vspace{-4mm}
\end{figure}

Como era de esperarse, con las configuraciones hechas a la medida de las aplicaciones, el estado de estas era \textit{Coherent}, sin la necesidad de ningún tipo de cambios en los dispositivos. Siendo así, como se ve en la figura \ref{fig:KillThemAll}, se matan los 3 contenedores desplegados, y se elimina uno de ellos.

\begin{figure}[H]
    \centering
    \caption{\\Se matan los contenedores con los servicios iniciales}
    \label{fig:KillThemAll}
    \includegraphics[width=0.9\linewidth]{images/Killing.png}
    \vspace{-4mm}
\end{figure}

La figura \ref{fig:FindAny}, muestra el como \textit{Bran}, define las órdenes para reiniciar los servicios para traer la aplicación a un estado nuevamente. En condiciones normales, se esperaría que esto sea lo único a realizar para adaptar la arquitectura, sin embargo, este no fue el caso.

\begin{figure}[ht]
    \centering
    \caption{\\Bran intenta reiniciar los contenedores}
    \label{fig:FindAny}
    \includegraphics[width=0.9\linewidth]{images/BranOrdersRestarts.png}
    \vspace{-4mm}
\end{figure}

\newpage

Ya que los contenedores no fueron desplegados, o ha sido modificados, de ninguna manera por \textit{DoThing}, no estos no están mapeados. Siendo así, como se ve en la figura \ref{fig:CantFindAny}, es necesario buscar los contenedores en la red, pero, al estar los servicios muertos, y depender de la consulta \textit{Http} para poder identificarlos, no puede encontrarlos.


\begin{figure}[ht]
    \centering
    \caption{\\DoThing no puede encontrar los contenedores}
    \label{fig:CantFindAny}
    \includegraphics[width=0.9\linewidth]{images/CantFind.png}
    \vspace{-4mm}
\end{figure}

Esto demuestra otra de las falencias con la implementación realizada, que es la búsqueda del crecimiento de la fase de conocimiento. El haber identificado los componentes, antes de que se presentaran problemas, hubiera posibilitado la ejecución de órdenes cuando los servicios estén caídos. Así mismo, se hubiera podido establecer una búsqueda que no dependiera del estado de los componentes, quizás usando tags.

El resultado final, es que \textit{Bran}, al no poder actuar de otra manera, como se observa en la figura \ref{fig:HadToAdd}, ordena acciones de adición para suplir los componentes. Esto, eventualmente, resulta en el regreso de la aplicación a un estado válido, pero sigue sin ser lo ideal.

\begin{figure}[ht]
    \centering
    \caption{\\Bran crea órdenes de adicción}
    \label{fig:HadToAdd}
    \includegraphics[width=0.9\linewidth]{images/ReLaunchIT.png}
    \vspace{-4mm}
\end{figure}